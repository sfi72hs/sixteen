
\documentclass[11pt,final]{article}
\usepackage[english]{babel}
%\usepackage[latin1]{inputenc}
\usepackage[utf8]{inputenc}
\usepackage{bbold}
\usepackage{amsmath}
%\usepackage{amssymb} % for \grtsym
\usepackage{comment}
\usepackage[left=2.cm,right=2.cm]{geometry}

\usepackage{color}
\usepackage{graphicx}
\usepackage{float}
\usepackage{caption}
\usepackage{subcaption}
\graphicspath{{./Images/}} % Directory in which figures are stored

%%%%%%%%%%%%%%%%%%%
\DeclareMathOperator*{\argmax}{arg\,max}
\newcommand{\argmin}{arg\,min}
\newcommand{\cyan}[1]{{\color{cyan}{#1}\normalcolor}}
\newcommand{\orange}[1]{{\color{orange}{#1}\normalcolor}}
\newcommand{\yellow}[1]{{\color{yellow}{#1}\normalcolor}}
\newcommand{\red}[1]{{\color{red}{#1}\normalcolor}}
\newcommand{\magenta}[1]{{\color{magenta}{#1}\normalcolor}}
\newcommand{\green}[1]{{\color{green}{#1}\normalcolor}}
\newcommand{\blue}[1]{{\color{blue}{#1}\normalcolor}}

\newcommand{\vv} {{\bm v}}
\newcommand{\be}{\begin{equation}} 
\newcommand{\ee}{\end{equation}}  
\newcommand{\bea}{\begin{eqnarray}}
\newcommand{\eea}{\end{eqnarray}}

\newcommand{\cD}{{\cal D}} 
\newcommand{\cL}{{\cal L}} 
\newcommand{\cM}{{\cal M}} 
\newcommand{\cN}{{\cal N}} 
\newcommand{\cO}{{\cal O}} 
\newcommand{\cU}{{\cal U}} 
\newcommand{\cW}{{\cal W}} 
\newcommand{\cZ}{{\cal Z}} 

\let\vaccent=\v % rename builtin command \v{} to \vaccent{}
\renewcommand{\v}[1]{\mathbf{#1}} % for vectors

%\newcommand{\vone} {{\bm 1}}
%\newcommand{\vv} {{\bm v}}
%\newcommand{\vw} {{\bm w}}
\newcommand{\ra}{\rightarrow}
\newcommand{\f}[2]{\dfrac{#1}{#2}}
\newcommand{\avg}[1]{\left\langle #1 \right\rangle} % for average
\let\setunion=\cup
\renewcommand{\cup}[1]{\left\{#1\right\}}
\let\oldt=\t
\renewcommand{\t}[1]{\tilde{#1}}
\newcommand{\w}{\omega}
\newcommand{\bb}[1]{\mathbf{#1}}
\newcommand{\s}{\sigma}
\newcommand{\tm}{\times}
\newcommand{\al}{\alpha}
\newcommand{\dl}{\Delta}
\renewcommand{\b}[1]{\bar{#1}}
\newcommand{\h}[1]{\hat{#1}}

\let\setunion=\cup
\renewcommand{\cup}[1]{\left\{#1\right\}}
\newcommand{\rup}[1]{\left(#1\right)}
\newcommand{\bup}[1]{\left[#1\right]}

\DeclareMathOperator{\arctanh}{arctanh}

\begin{document}

\section*{Beneficial epidemic transmission with link generation.}

\subsection*{The model}
Consider a network as a set of nodes with a variable attached on them defining their epidemic state. A node can either be infected, $I$, or susceptible, $S$. We consider the SIS model where there can be transitions from any state to any other state. These transitions are regulated by two parameters:
\begin{itemize}
\item an infection transmission rate $\beta$ if a susceptible node is infected by an infected node; 
\item a recovery rate $r$ for an infected node to become susceptible.
\item In addition, we assume that an infected node will generate $\Delta$ new links upon infection, these can be targeted to any node, regardless their state;
\item $\al$ represents the probability that, in creating new links, an infected nodes selects susceptible nodes ($\al=0$), randomly ($\al=1$) or infected nodes ($\al>1$).
\end{itemize}

Define $S$ and $I$ as the ratio of susceptible and infected nodes respectively; by definition $S+I=1$. Also, $[SI]$ is the number of $S \ra I$ edges, and so on for the other remaining three combinations.
The differential equations governing this process are:
\bea
\dot{I}&=& \beta [SI] -rI \\
\dot{[SS]}&=& -\beta \,[SI]\, 2\f{[SS]}{S} +r [SI] \bup{\f{k_{0}+I \Delta}{k_{0}+I \Delta +\Delta}} \\
\dot{[SI]}&=& \beta[SI] \rup{ 2\f{[SS]}{S} -\f{[SI]}{S}-1} -r[SI] +\beta[SI]\dl \blue{\f{S}{S+I\al} }+2\,r\,[II]  \bup{\f{k_{0}+I \Delta}{k_{0}+I \Delta +\Delta}} \\
\dot{[II]}&=& \beta[SI] \rup{\f{[SI]}{S} +1} +\beta\,[SI]\, \dl  \blue{\f{I\al}{S+I\al}} -2r [II]
\eea

These equations are complemented with the average degrees of an infected node $\avg{k_{I}}$, of a susceptible node $\avg{k_{S}}$ and the total expected number of links $E$:

\bea
\avg{k_{I}}&=& k_{0}+\dl +I\dl\\
\avg{k_{S}}&=& k_{0}+I \dl \\
E&=&\f{k_{0}}{2} +I\dl = [SS]+[II]+2[SI]
\eea
where $k_{0}$ would be the average degree in case of no link generation. 

A variant of the model considers the case where an infected node keeps generating $\dl$ new edges at each time step, not only at infection time. In this case in the last two equations one replaces the term $[SI]$ to $I$:
\bea
\dot{[SI]}&=& \beta[SI] \rup{ 2\f{[SS]}{S} -\f{[SI]}{S}-1} -r[SI] +\beta \,\red{I}\dl  \f{S}{S+I\al} +2\,r\,[II]  \bup{\f{k_{0}+I \Delta}{k_{0}+I \Delta +\Delta}} \\
\dot{[II]}&=& \beta[SI] \rup{\f{[SI]}{S} +1} +\beta\,\red{I}\, \dl \f{I\al}{S+I\al} -2r [II]
\eea

\bibliography{bibliography}{}
\bibliographystyle{unsrt}



\end{document}