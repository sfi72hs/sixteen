\documentclass[11pt,final]{article}
\usepackage{amsmath}
\usepackage{amssymb}

%%%%%%%%%%%%%%%%%%%

\begin{document}

\section*{Beneficial epidemic transmission with link generation}

\subsection*{The model}
Consider a network as a set of nodes with a variable attached on them defining their epidemic state. A node can either be infected, $I$, or susceptible, $S$. We consider the SIS model where there can be transitions from any state to any other state. These transitions are regulated by two parameters:
\begin{itemize}
\item an infection transmission rate $\beta$ for a susceptible node to be infected by a neighboring infected node; 
\item a recovery rate $r$ for an infected node to become susceptible.
\item In the absence of infected nodes, we assume that the average degree of a node is $k_0$.
\item In addition, we assume that an infected node will generate $\Delta$ new links upon infection, these can be made to any node of either state;
\item $\alpha$ represents the bias for what type of node is linked to by the new links. Either susceptible nodes are always selected ($\alpha=0$), susceptible
    nodes are selected with preference ($0<\alpha<1$), all nodes are selected uniformly ($\alpha=1$), or infected nodes are selected with preference ($\alpha>1$).
\item If an infected node recovers, it loses $\Delta$ edges at random.
\end{itemize}

Define $S$ and $I$ as the ratio of susceptible and infected nodes respectively to the total population.
By definition $S+I=1$. Let $[SI]$ be the number of $S \to I$ edges normalized by the total population size,
and so on for the other two combinations.
The differential equations governing this process are:
\begin{equation}
    \begin{aligned}
\dot{I}&=-\dot{S}= \beta [SI] -rI \\
\dot{[SS]}&= -\beta \,[SI]\, \frac{2[SS]}{S} +r [SI] \left(\frac{k_{0}+I \Delta}{k_{0}+I \Delta +\Delta}\right) \\
\dot{[SI]}&= \beta[SI] \left( 2\frac{[SS]}{S} -\frac{[SI]}{S}-1 \right) -r[SI] +\beta[SI]\Delta \frac{S}{S+I\alpha}+2\,r\,[II]  \left(\frac{k_{0}+I \Delta}{k_{0}+I \Delta +\Delta}\right) \\
\dot{[II]}&= \beta[SI] \left(\frac{[SI]}{S} +1\right) +\beta\,[SI]\, \Delta \frac{I\alpha}{S+I\alpha} -2r [II]
    \end{aligned}
\end{equation}
We can reduce this system to only three variables by using the following equations for node and edge total
\begin{equation}\label{eq:closure}
    \begin{aligned}
S + I &= 1\\
[SS]+[SI]+[II] &= \tfrac12 k_0 + I\Delta
    \end{aligned}
\end{equation}

The average degrees of an infected node $\langle k_{I}\rangle$ and of susceptible node $\langle k_{S}\rangle$ are
\begin{equation}
    \begin{aligned}
\langle k_{I}\rangle&= k_{0}+\Delta +I\Delta\\
\langle k_{S}\rangle&= k_{0}+I \Delta
    \end{aligned}
\end{equation}

We also consider a variant where an infected node keeps generating $\Delta$ new edges per unit time,
not only at infection time. In this variant we do not consider recovery of an infected node.
\begin{equation}
    \begin{aligned}
\dot{I}=-\dot{S}&= \beta [SI] \\
\dot{[SS]}&= -\beta \,[SI]\, 2\frac{[SS]}{S} \\
\dot{[SI]}&= \beta[SI] \left( 2\frac{[SS]}{S} -\frac{[SI]}{S}-1 \right) -r[SI] + I \Delta \frac{S}{S+I\alpha} \\
\dot{[II]}&= \beta[SI] \left( \frac{[SI]}{S} +1 \right) +I \Delta \frac{I\alpha}{S+I\alpha}
    \end{aligned}
\end{equation}
In this variant, we do not have an equation for the total number of edges and we cannot reduce the system to only three variables.

\bibliography{bibliography}{}
\bibliographystyle{unsrt}



\end{document}
